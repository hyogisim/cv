\section{CURRENT PROJECTS}
\vspace{.03in}
%%%%%%%%%%%%%%%%%%%%%%%%%%%%%%%%%%%%%%%%%%%%%%%%%%%%%%%%%%%%%%
  {\bf Workflow-Aware Multi-Tiered Storage System in High Performance Computing}
   {\it \footnotesize --- 2016-present, Oak Ridge National Lab.}
   \begin{itemize}[leftmargin=*]
    \setlength\itemsep{-0.02in}
%    \item[-] Designing a software-defined storage system that manages data files
%	     in heterogeneous storage tiers based on user-specified workflow execution policies.
    \item[-] Designing and developing a NoSQL store-based global metadata manager
	     that provides a unified namespace abstraction atop heterogeneous
	     storage tiers.
    \item[-] Developing a client-side shared library that allows applications to interact
             with the metadata manager.
    \item[] {\small(FUSE, Lustre, Ceph, MySQL, HyperDex)}
   \end{itemize}
%%%%%%%%%%%%%%%%%%%%%%%%%%%%%%%%%%%%%%%%%%%%%%%%%%%%%%%%%%%%%%
  \vspace{-0.15in}
  {\bf Development of a Checkpoint File System for HPC Burst Buffers}
   {\it \footnotesize --- 2017-present, Oak Ridge National Lab.}
   \begin{itemize}[leftmargin=*]
    \setlength\itemsep{-0.02in}
    \item[-] Designing and developing an ephemeral, distributed file system for
	     node-local burst buffers to facilitate checkpointing of HPC applications.
    \item[-] Developing a scalable mechanism to manage the file system metadata
             using a NoSQL database for supporting shared checkpoint files.
    \item[] {\small(FUSE, MPI, MDHIM, LevelDB)}
   \end{itemize}
%%%%%%%%%%%%%%%%%%%%%%%%%%%%%%%%%%%%%%%%%%%%%%%%%%%%%%%%%%%%%%
\begin{comment}
  \vspace{-0.15in}
  {\bf Development of a Fast Temporary Storage for Data-Intensive Applications using NVMe SSDs}
   {\it \footnotesize --- 2016-present, Oak Ridge National Lab., Virginia Tech}
   \begin{itemize}[leftmargin=*]
    \setlength\itemsep{-0.02in}
    \item[-] Designing a zero-copy, out-of-core framework to temporally persist
             application objects in volatile memory.
    \item[-] Implementing a userspace layer that directly manages NVMe
             devices to bypass the kernel page cache and file system.
    \item[-] Identifying a potential extension of the NVMe protocol 
             for offloading physical space management to the device and
             providing an object-based interface to applications.
    \item[] {\small(Linux, NVMe, NVMeDirect)}
   \end{itemize}
\end{comment}
 
