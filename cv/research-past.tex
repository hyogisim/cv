\section{DEVELOPMENT EXPERIENCES ON FILE \& STORAGE SYSTEMS} 
\vspace{0.03in}
%%%%%%%%%%%%%%%%%%%%%%%%%%%%%%%%%%%%%%%%%%%%%%%%%%%%%%%%%%%%%%
  %\vspace{-0.15in}
  {\bf File System-Integrated Search and Discovery Services for HPC}
   {\it \footnotesize --- 2015-2016, Oak Ridge National Lab. and Virginia Tech}
   \begin{itemize}[leftmargin=*]
    \setlength\itemsep{-0.02in}
    \item[-] Designed and developed a file system-integrated metadata indexing framework
	     that supports the user-defined tagging in GlusterFS.
    \item[-] Developed a command-line utility using the GlusterFS API to
             allow interactive, tagging-based file search queries in SQL syntax.
    \item[-] Developed a computation offloading framework via a tagging-based file search,
             similar to `\fixed{find -exec}'.
    \item[-] Developed an automatic metadata
	     extraction framework based on the tagging-based file search.
    %\item[-] Implemented the prototype framework in GlusterFS and command-line utilities.
    \item[] {\small(GlusterFS, SQLite, Linux)}
   \end{itemize}
%%%%%%%%%%%%%%%%%%%%%%%%%%%%%%%%%%%%%%%%%%%%%%%%%%%%%%%%%%%%%%
  \vspace{-0.15in}
  {\bf Analysis-Aware Storage System for HPC}
    {\it \footnotesize --- 2013-2015, Oak Ridge National Lab. and Virginia Tech}
    \begin{itemize}[leftmargin=*]
    \setlength\itemsep{-0.02in}
    \item[-] Designed an active execution framework based on SCSI T10 OSD-2 specification
             and implemented the extended OSD-2 protocol the on the Linux TGT.
    \item[-] Extended the host-side OSD initiator driver and exofs in Linux Kernel
             to support the active execution framework.
    \item[-] Designed and developed a FUSE file system that manages
             the array of active OSD devices.
    \item[-] Integrated a workflow manager within the file system to support scientific
             workflow processing across the array of active OSD devices.
    \item[-] Developed a provenance management framework
             with a light-weight database within the FUSE file system
             based on the OSD-2 object abstraction
    \item[] {\small(Linux Kernel, SCSI T10 OSD-2 Protocol, Linux TGT, FUSE, SQLite)}
    \end{itemize}
%%%%%%%%%%%%%%%%%%%%%%%%%%%%%%%%%%%%%%%%%%%%%%%%%%%%%%%%%%%%%%
  \vspace{-0.15in}
  %{\bf Managing Multimedia Data for Content Servers with Hybrid Storage Architecture}
  {\bf Hierarchical Data Management in Media Servers with Hybrid Storage Architecture}
    {\it \footnotesize --- 2007-2009, Database Lab., Hanyang University}
    \begin{itemize}[leftmargin=*]
    \setlength\itemsep{-0.02in}
    \item[-] Developed a content popularity analyzer that periodically analyzes 
             request log files and ranks highly requested contents
             in a commercial media server running Windows Media Server.
    \item[-] Developed a light-weight file system based on ext2
             for the caching servers equipped with 
             storage class memory devices, e.g., PRAM.
    \item[] {\small(Linux Kernel, Windows Media Server)}
    \end{itemize}
%%%%%%%%%%%%%%%%%%%%%%%%%%%%%%%%%%%%%%%%%%%%%%%%%%%%%%%%%%%%%%
  \vspace{-0.15in}
  {\bf Development of a NAND Flash Memory-Based File System Supporting Transaction and
      Record Structure} 
    {\it \footnotesize --- 2006-2008, Database Lab., Hanyang University}
    \begin{itemize}[leftmargin=*]
    \setlength\itemsep{-0.02in}
    \item[-] Developed a DBMS that integrates FTL and directly manages a raw NAND flash
             memory chip via Linux MTD layer for managing EPG (Electronic Program Guide)
             data in a set top box.
    \item[-] Developed a flash-aware buffer management policy in PostgreSQL
             to generates a flash-friendly LBA sequence for NAND flash memory.
    \item[-] Developed a software framework that generates a set of pre-defined I/O requests
	     and identifies a FTL mapping algorithm, e.g., page- or block-level mapping,
             of a NAND flash memory-based
             storage device.
    \item[] {\small(Linux Kernel, Linux MTD, Wisconsin Storage System, PostgreSQL,
             ARM-based embedded board)}
    \end{itemize}
  %\vspace{-0.15in}

